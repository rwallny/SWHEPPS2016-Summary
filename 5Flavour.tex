%\newpage
\section{Session on Heavy Flavour Physics}\label{flavour}{\it editors: K. Kirch, O. Schneider, A. Signer}
%\fix{Preliminary version}

The Heavy Flavour session consisted of the following three experimental presentations:
\begin{itemize} \setlength{\itemsep}{-1ex}
\item ``Lepton violation'', by Patrick Koppenburg (Nikhef, Amsterdam);
\item ``Rare decays at LHCb'', by Nicola Serra (University of Zurich), delivered by Patrick Koppenburg;
\item ``CP violation and CKM physics'', by Fred Blanc (EPFL, Lausanne).
\end{itemize}

\noindent The speakers were given the task to review their sub-fields ({\it i.e.},  lepton flavour violation and lepton flavour 
universality, rare decays and the V-A structure of weak interactions, CP violation and CKM measurements in $b$-, $c$-,
and kaon physics) in a global manner, and focus on measurements which are of most relevance for the discussions of
future strategies in high-energy particle physics in Switzerland. With various emphases, the speakers underlined the
significant experimental progress achieved during the past several years in the search for New Physics in heavy-flavour
decays, explained the current ``flavour anomalies'' that have appeared (and sometimes strengthened) in the past few 
years both at the $B$ factory experiments (Babar and Belle) and at LHCb with Run~1 data, and discussed the motivation 
and prospects of already planned upgrades at LHCb  and Belle II.

\medskip

\noindent An emblematic check of the electron-muon universality in $b$ decays is through the measurement of the ratio $R_K = 
{\cal B}(B^0 \to K^{*0}\mu^+\mu^-)/ {\cal B}(B^0 \to K^{*0}e^+e^-)$ performed in the low 
$q^2 (= m_{\ell^+\ell^-}^2$) region, where the  theoretical prediction is the cleanest. The LHCb measurement falls  
$2.6\,\sigma$ below the Standard Model (SM) value of one, 
and is consistent with the statistically less powerful results of BaBar 
and Belle. On the other hand, results involving the tau lepton from semileptonic $B$ decays exhibit a much more
significant effect: the Babar, Belle and LHCb measurements of 
$R(D) = {\cal B}(B \to D\tau^+\nu_\tau)/ {\cal B}(B \to D\ell^+\nu_\ell)$ and
$R(D^*) = {\cal B}(B \to D^*\tau^+\nu_\tau)/ {\cal B}(B \to D^*\ell^+\nu_\ell)$ ($\ell = e, \mu$) all consistently lie above 
the SM predictions, with a two-dimensional average presently providing a clear $4.0\,\sigma$ evidence of 
departure from lepton universality.
%
Because non universality implies lepton flavour violation in many New Physics models, these experimental anomalies
reinforce the motivation of the searches for decays where lepton numbers are violated. 
Decays such as $B^0_{(s)} \to e^\pm\mu^\mp$, $B^+\to \mu^+\mu^+h^-$ and $D^0 \to e^\pm\mu^\mp$, where LHCb 
has the leading sensitivity, have been looked for but not seen so far. Lepton-flavour violating $\tau$ decays, where 
the $B$ factories are leading but not signal is observed either, are also very promising.
\medskip

\noindent In the rare decay sector, the joint observation of $B^0_s \to \mu^+\mu^-$ by CMS and LHCb, as well as the ATLAS result, set very strong constraints on models beyond the SM, basically ruling out SUSY with large $\tan(\beta)$ values, and more generally excluding large non-SM contributions to the Wilson coefficient $C_{10}$. The loop-suppressed decay $B^0 \to K^{*0}\mu^+\mu^-$, $K^{*0}\to K^-\pi^+$, is attracting a lot of theoretical and experimental interest. The kinematics of the four-body final state can be described in terms of $q^2 = m_{\mu^+\mu^-}^2$ and three helicity angles. A full angular analysis offers many $q^2$-dependent observables with clean SM predictions. For the first time LHCb has measured a complete set of observables for both $B^0$ and $\bar{B}^0$ decays. All observables are consistent with SM predictions, except one of them (called $P'_5$) in the low $q^2$ region, causing a fit to the $C_9$ Wilson to deviate by  $3.4\,\sigma$ from the SM. The Belle data also displays a discrepancy in $P'_5$, in agreement with the more precise LHCb data. Interestingly, many exclusive $b\to s \mu^+\mu^-$ decays, such as $B^+ \to K^{(*)+}\mu^+\mu^-$,  $B^0 \to K^0\mu^+\mu^-$, $B^0_s \to\phi \mu^+\mu^-$ and 
$\Lambda_b \to \Lambda  \mu^+\mu^-$ are found to have a differential branching fraction  consistently smaller than the SM prediction in the low $q^2$ region; this is perhaps related to the ``muon deficit'' seen in the $R_K$ observable mentioned above.

\medskip
\noindent Radiative $B$ decays can be used to test the V-A structure of weak interactions, through the measurement of the photon
polarization which is sensitive to right-handed currents (Wilson coefficient $C'_7$). First results from LHCb are available
using $B^0 \to K^{*0}e^+e^-$ (at $q^2\to 0$), $B^+ \to K^+\pi^-\pi^+\gamma$ (exploiting three-body hadronic system) and 
$B^0_s \to \phi\gamma$ (time-dependence). All are consistent with the SM, but will become constraining only 
with more data collected at Run 2 and beyond.
\medskip

\noindent Measurements of CP violation and CKM observables got a big boost with LHC data. Notable advances are the measurements of the $B^0_s$-mixing induced phase $\phi_s$ in exclusive $b \to c \bar{c}s$ transitions, such as $B^0_s \to J/\psi \phi$ from ATLAS, CMS and LHCb, and other modes from LHCb. The current determination of $\phi_s$ is consistent with the SM prediction. The precision on the CKM angle $\gamma$, which can be determined from $B\to DK$ tree decays only and hence constitutes an important reference in the precision test of the consistency of the CKM picture, is improving, with the current determination from LHCb being twice as more precise as that of Babar or Belle.  A longstanding tension in this picture is that due to a $3\,\sigma$ inconsistency between $B$-factory determinations 
of $|V_{ub}|$ performed with exclusive and  inclusive $B$ decays. A new LHCb measurement with baryons (exclusive $\Lambda_b \to p \mu^-\bar{\nu}_\mu$), which has a different sensitivity to a possible enhancement of the right-handed contribution to the weak current, brings interesting new information but shows that the experimental results 
cannot be reconciled with such an enhancement.

\medskip

\noindent In summary, a few $3-4\,\sigma$ deviations from SM expectations are seen in the heavy flavour sector, which could
be due to New Physics or QCD effects. All the measurements are still dominated by statistical uncertainty, meaning that
much more can be learned with more data. However, the timescale for settling these anomalies is much longer than
needed for the 750~GeV bump seen in 2015 by the ATLAS and CMS experiments. In the immediate future, Run 2 data
will multiply LHCb's heavy flavour statistics by at least four. 
Then the LHCb experiment will be upgraded to collect an order of magnitude more data. 
In parallel the  Belle II experiment at Super-KEKB, which has very much complementary physics reach, 
will ramp up and shoot for 50 times the current Belle statistics.  Switzerland is involved in this endeavour through the LHC experiments, mainly LHCb where EPFL and UZH have strong commitments including for the detector upgrade.