\section{Session on Higgs Boson Physics}\label{higgs}{\it editors: F. Canelli, M. Weber}
%\fix{editors: Florencia, Michele}

\noindent The Higgs boson physics session consisted of the following two experimental and one experimental presentations:
\begin{itemize} \setlength{\itemsep}{-1ex}
\item ``SM Higgs properties'', by Sinead Farrington (Univ. of Warwick, UK);
\item ``SM Higgs couplings and BSM Higgs'', by Mauro Donega (ETH Zurich);
\item ``Theory tools for precise EW and Higgs physics'', by Stefano Pozzorini (University of Zurich).
\end{itemize}

\noindent One of the primary goals of the LHC is to probe the nature of Electroweak Gauge 
Symmetry breaking. In the standard model (SM), electroweak symmetry
breaking occurs as a result of the Higgs mechanism, which predicts an
additional scalar field, accompanied by an associated scalar boson. A
Higgs boson was discovered by the ATLAS and CMS experiments in the
Summer of 2012, with properties consistent with SM predictions.
Understanding the properties of this new state is of fundamental
importance and requires further investigation in the form of a precision
experimental program. Any deviation in the predicted properties of the
Higgs boson is a strong, unambiguous signature for new physics. 
\medskip

\noindent Since its observation, measurements in all of the main production and decay
channels of the Higgs boson have been completed with the full integrated
luminosity available from LHC Run 1, which consists of 5\fb  at $\sqrt{s} = 7$
TeV from 2011, and 20\fb at  $\sqrt{s} =$  8 TeV from 2012. 
Measurements of the mass, total width, spin, couplings, CP mixtures, as well as searches for
multiple Higgs bosons represents the main legacy of the LHC Run 1. Some
of the primary results are: 

\begin{itemize}
\item The total Higgs boson cross-section has been measured to be $1.09 \pm 0.11$ of the SM prediction.
\item Production modes via gluon fusion and vector-boson fusion have been observed. 
\item Most decay modes have been observed: $\gamma \gamma$, ZZ, WW, and $\tau\tau$.
\item The mass has been determined with 0.2\% precision,  $m_H$ = 125.0 $\pm$ 0.21 (stat) $\pm 0.11$ (syst).
\item The spin and parity are consistent with spin 0 and even parity and exclude non-SM scenarios at 99.9\% C.L.
\item Constraints on the width have been set to $\sim 4 \times$  SM Higgs boson width. 
\item Coupling properties to fermions and bosons are only established at the 10-20\% level.
\end{itemize}

The picture of the minimal standard model may now appear complete,
yet open questions remain on its stage, which will be addressed by the
current and future LHC runs. The LHC Run 2 has successfully started in
2015, opening a new period of particle physics exploration, at higher
energy and intensity. Higgs physics results shown at ICHEP 2016 based on
up to 20\fb of 13 TeV collisions now achieve similar sensitivity to
that of the previous Run 1 data. These data are allowing an increase in
precision and opening up new channels that will undoubtedly deliver more
insight on the electroweak model, its symmetry-breaking mechanism. In
particular, the fermionic Higgs sector is relatively unknown, and will
be probed in detail. Future upgraded detectors and two orders of
magnitude more statistics will allow for a precision picture of the
symmetry-breaking mechanism and possible solutions to its conundrums.
\medskip

\noindent The discovery of a Higgs boson opens up the question of whether the
Higgs boson observed is the SM Higgs boson or one of several Higgs
boson, predicted by extensions of the SM. There are strong theoretical
arguments that suggest nature should have an extended Higgs sector with
additional charged or neutral (CP-even or odd) Higgs bosons (2HDM
theories), or that the Higgs boson is not an elementary particle,
introducing a new strong interaction at energies beyond the TeV scale
(little or composite Higgs theories). These theories are continually
probed with the greater reach in Run 2 and HL-LHC. Complementary
searches for anomalous or forbidden decays of the Higgs boson, including
invisible decays, are also an important part of the physics program of
the LHC. In many BSM scenarios, the couplings of the Higgs boson are
expected to show discrepancies with respect to the prediction of the SM.
During Run 1 the overall branching fraction of the Higgs boson into BSM
decays was determined to be less than 34\% at 95\% C.L. leaving ample
space for new physics. Significant improved precision and sensitivity is
expected to be achieved in the future LHC analyses.