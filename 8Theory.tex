\section{Session on Theory, Connection to other Fields}\label{theory}

\fix{Editors: Gino with help from  Riccardo and Thomas}


The session consisted of the following four presentations: 
\begin{itemize} \setlength{\itemsep}{-1ex}
\item {\em Beyond the SM: the big picture},  by  Riccardo Baribieri (SNS Pisa \& ITS-ETH); 
\item {\em  The flavor problem}, Gino Isidori (UZH); 
\item {\em  Perspectives for precision calculations}, Babis Anastasiou (ETH); 
\item {\em  Connections between cosmology and high-energy physics}, Andrey Katz (Geneve).
\end{itemize}

Riccardo Barbieri presented a general discussion about the main open problem in particle physics. 
He  discussed various arguments why the SM cannot be considered a complete theory,
emphasising in particular:  i) the (electroweak) hierarchy problem (or the instability of the Higgs mass term,
with respect to quantum corrections from high-energy degrees of freedom); 
ii) the flavour puzzle (or the lack of understanding for the large, 
and apparently non accidental,  span of the entries of the SM Yukawa couplings); 
iii) the non-vanishing values of neutrino masses, that unambiguously 
signals the presence of physics beyond the SM;
iv)  the lack  of an explanation, within the SM, for the quantization of the U(1) charges; 
v) the lack of understanding, within the SM, for the vanishing of CP 
violation in the QCD Lagrangian (that would naturally point to the presence of an axion).
He presented some possible solutions to the above problems, some of which involving 
NP within the direct reach of the LHC, other with observables effects only via precision
tests, and others with observables effects only at the cosmological level. 
He stressed  that the way to extend this highly successful theory 
at high energies is, at present, very uncertain.  This fact, and the very nature of Particle Physics,
call for highly diverse frontiers of research, involving both low- and high-energy experiments
at particle accelerators and beyond.

The presentation of Gino Isidori was focused on flavour physics. He emphasised, and 
demonstrated with a few examples, that flavour physics represents
a very powerful tool for indirect searches of new physics. This is particularly true in the 
present scenario of large uncertainty about the nature of physics beyond the SM. 
He also stressed the strong connections of flavour physics (and, more generally, low-energy experiments) 
with the other ``frontiers'' of particle physics (in particular neutrino physics and high-pT physics).
Finally, he emphasised that recent data have helped us to identify a very rich ``new frontier''
within flavour physics, namely the study of Lepton Flavor non Universality, whose interest 
was not properly recognised in the past. 

Babis Anastasiou outlined the importance of precise theoretical calculations in high-energy 
physics. He emphasised that the combination of high-precision theory plus 
high-precision experiments is the recipe for making progress in the field: 
one cannot make progress without combining these two ingredients. He also briefly reported about the great progress
achieved in the last few years on precise calculations in 
perturbative QCD (relevant for collider physics): NNLO calculations are reaching 
the maturing level, while the N3LO level has just been started. 
Finally, he stressed that Switzerland is at the very frontier of this filed of research,
 and that precision phenomenology requires a long standing and stable support to flourish.
 
 Andrey Katz presetned the so-called ``Hidden Valley"  scenario as a general
framework to highlight the connections between cosmology and high-energy physics.
 The basic idea of this class of models is that there can be low-scale new particles
  (e.g. with masses around or below 1~GeV), that couple to the SM very weakly, 
  due to the exchange of very heavy (e.g.~1 TeV) new states. With a few examples he 
  emphasised that this scenario can be a building block of models addressing 
Dark Matter and the matter-antimatter asymmetry, and where  
future cosmological observations might have direct implication for future LHC searches.



