%\newpage
\section*{Day 2 Summary: Flavour and Low Energy Summary and Connection to other 
fields}\label{discussionconnetion}{\it discussion leaders: G. Isidori, O. Schneider}

\noindent The discussion covered three main subjects: I) the prospects for indirect NP searches, mainly via flavour-physics 
measurements, within the LHC experiments (LHCb, but also ATLAS and CMS);
II) the connections between low- (non-LHC) and high-energy physics experiments within the first pillar, with particular attention to the role of PSI;
III) the connections between the first pillar and the other two pillars.
The main points of discussions on these three main subjects can be summarised as follows.

\begin{enumerate}
\item[I.a] {\em Flavour physics and indirect NP searches at ATLAS and CMS.} 
The question has been raised if ATLAS and CMS should consider possible 
modifications of their upgrade plans (for the HL phase) in view of present flavour anomalies,
especially in the absence of direct signals of NP. These anomalies, and also various theoretical 
arguments, seem to suggest a particular interest in tau- and $b$-quark enriched final states
and, more generally, in an optimal flavour-tagging (and flavour-discrimination) efficiency. 
After an extensive discussion, a consensus was reach on the fact that this request
is already well addressed by the present upgrade plans of ATLAS and CMS,
taking into account also the fact that these high-$p_{\rm T}$ experiments 
must optimize the NP sensitivity in all possible directions.

\item[I.b] {\em  LHCb upgrade}. 
Motivated by the strong interest in an extension of the $b$-physics programme at the LHC,
advocated in various theory talks at the meeting and reinforced by the recent interesting results in this sector, 
the possibility of a further upgrade of the LHCb experiment has been discussed.
The already planned LHCb upgrade aims at collecting $50~{\rm fb}^{-1}$ by 2030.
In principle, a luminosity $\sim25$ times higher 
% LHCb upgrade aims at a luminosity of 2e33 cm-2s-1
% HL-LHC aims at a peak luminosity of 5e34 cm-2s-1
% ==> the ratio is 5e34/2e33 = 25
would be available in the HL phase of the LHC. Can a further-upgraded LHCb stand such a luminosity (or a significant fraction of it)? 
Beside the maximal luminosity, can some of the present LHCb performances (e.g.\ on electron and tau modes) 
be increased in view of a further upgrade? 
The LHCb collaboration is considering this interesting option, but a detailed answer on its feasibility requires time.

\item[II] {\em Low-energy physics}. As outlined in the theory talks, the indirect NP searches performed at low-energies 
are extremely interesting and, to a large extent, independent from the direct searches performed at high energies. 
This point was further emphasized during the discussion session, with particular attention to the PSI programme. 
In particular, it was stressed that it is important to secure (both in terms of funding and manpower) 
the interesting and ambitious $\mu \to 3 e$ phase-II programme, independently of the developments 
at the high-energy frontier. The point was raised that a potential firm evidence of lepton-flavour non-universality
in $B$ decays would, on general grounds, render the physics case of  CLFV searches 
in $\mu$ decays even stronger.  However, it was also concluded that the opposite is not true 
(CLFV in  $\mu$  decays could occur independently of LFU in $B$ decays), since the connections between these 
two sectors is very model dependent.

\item[III.a] {\em Connections with the neutrino programme.} 
The main issue discussed has been the possible interplay (or better the influence and the possible synergies) between
the experimental programme at the high-energy frontier and the accelerator-based neutrino programme. 
While the ultimate physics goals of the two programmes are certainly connected, it has been concluded that the two programmes 
run essentially in parallel: the results of the former have very little influence on the latter, and vice-versa. On the other hand,
possible synergies can be envisaged at a technical level. 

\item[III.b] {\em Connections with the DM programme (direct \& indirect DM searches).} 
Similarly,  the interplay between the searches for DM candidates at colliders 
and the direct and indirect searches of DM, performed in underground laboratories 
or via astrophysical data,  has been discussed. 
In this case there are certainly strong connections;  however, these are very model dependent
(being determined by unknown physics beyond the SM). 
At present is not possible to determine a clear influence of the results of 
direct and indirect searches of DM on the HEP programme. The situation may change in view of a positive evidence of physics beyond the SM. 
The importance of combined analyses of both high-energy data, underground searches, and astrophysical data,
in case of a positive signal, has also been stressed.

\end{enumerate}
