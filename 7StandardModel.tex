\section{Session on Standard Model Physics}\label{standardmodel}{\it editors: F. Canelli, M. Weber}

\noindent The Standard Model physics session  consisted of the following two experimental and one experimental presentations:
\begin{itemize} \setlength{\itemsep}{-1ex}
\item ``Experimental overview of the EW and QCD'', by Kostas Theofilatos  (ETH Zurich);
\item ``Top quark physics results and prospects'', by Richard Hawkings (CERN);
\item ``Motivations for future precision studies of EW and Higgs physics'', by Francesco Riva (CERN).
\end{itemize}

\noindent The Standard Model (SM) has been developed and tested in the last decades 
through dedicated theoretical and experimental research to a very high level of accuracy.
A robust program for precision measurements of the SM remains of key
importance to the LHC. The physics of the SM and its rich phenomenology
at the highest energy scale continues to be of prime scientific value.
Moreover, deviations of precision measurements to theoretical
expectations can probe physics beyond SM where new physics is not
accessible at tree-level. A thorough understanding of the SM processes
is key for understanding the background to possible discoveries. A
summary of the SM measurements and expectations for results at the LHC
is given below, as presented at the SWHEPPS workshop.

\subsubsection*{The Electro-Weak (EW) interaction}

\noindent With the availability of the Higgs Boson mass
measurement the global EW fit becomes over-constraint. The global fit
prefers a slight lighter Higgs and although no evidence for an overall
inconsistency is found, there are a number of tensions in the fit. In
particular, the lepton asymmetry and the b forward backward asymmetry
with a p-value of 0.2\%. Measurements from the LHC have currently
typical precisions at the level of 0.5\%, and will be further improved
to reach the precision similar to SLD and LEP (0.1\% ) and the Tevatron
(0.2\%). The measurement of the mass of the W Boson is a challenging
example for a hadron collider. Special experimental (recoil, special
runs, calibrations) measures have to be addressed as well as theoretical
and PDF uncertainties considered. The expectation is that an uncertainty
in the W Boson mass of order 10~MeV can be reached. Triple gauge and
quartic gauge interactions are also important tests of the gauge
structure. Strong limits on anomalous couplings have been set. Many
multi-boson channels have been observed, but not all, and the precision
frontier is yet to be reached. Thus more statistics and measurements are
needed. Especially ratios between di-boson (including photons)
production cross sections as well as new statistical techniques based on
machine learning will enable to reach precision measurements in this
field. 

\subsubsection*{The Strong interaction (QCD)}

\noindent Among the main observables to test
the QCD asymptotic freedom ($\alpha_s$) and advance the knowledge of PDFs
are the jet transverse momentum and jet multiplicity spectra measured at
the LHC (with jet of pt up to 3 TeV). Photon as well as
Vector-Boson+jets differential cross sections also provide input to PDFs
and $\alpha_s$. This is a very active field, including efforts on mixed
QCD-EW corrections in the matrix element and parton shower Monte Carlo
simulation codes. Electro-Weak precision results at the LHC imply
excellent understand of QCD.

\subsubsection*{The top quark}

\noindent The top quark plays a prominent
role as the most massive elementary particle, with a Yukawa coupling
almost exactly unity. Events with top quarks are copiously produced at
the LHC and thus it represents a laboratory for SM studies at the
highest energies. It also represents one of the most important
backgrounds in searches for new physics involving new heavy states. The
phenomenology of the top quark is very well known for production and
decay modes. Inclusive and differential cross section measurements are
available, with experimental and theoretical systematic uncertainties
around 4\%. The uncertainty is yet slightly larger in Run-2, dominated
by luminosity and modeling uncertainties, but significant improvements
are expected as more data is collected. The differential cross sections
as well as top quark production with heavy flavor jets or Vector Bosons
probe in detailed the description of the kinematics in the theory
calculations (up to NNLO) and MC simulation codes and provide inputs for
their tuning, which is critical for many of the searches performed at
the LHC. A difficulty to describe the transverse momentum of the top
quarks persists and needs to be addressed. Also for boosted topologies,
where the top quark decay products merge in singe reconstructed objects,
it is important to improve the modelling to fully exploit the jet
substructure. These are particularly important at 13 TeV and 14 TeV
center of mass energy. For the electroweak production of top quarks, the
measurements are generally in agreement with NLO+NNLL prediction with an
experimental uncertainty of 9\% to 14\% with a theoretical precision of
about 5\%. Thus improvements are expected from the larger Run-2 and
future data sets and from more differential measurements that allow for
precise measurements of the couplings. As for the mass of the top quark,
several measurement and analyses are performed, utilizing all decay
modes of the top quark. The relative precision of the mass measurements
is below 0.5\%, however, global electro-weak fits need estimations of
the pole mass, which is not directly accessible with the traditional
analyses. Several alternate mass measurements are pursued (e.g.~from the
cross section or from m(ttj)), but the precision of these is still far
behind the direct reconstruction techniques, thus representing a main
effort in the future.