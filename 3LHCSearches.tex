\section{Session on Searches for BSM Physics at the LHC}\label{lhcbsm}{\it editors: B. Kilminster, T. Golling}

\noindent The session on searches for BSM physics at the LHC consisted of the following two experimental and one experimental presentations:
\begin{itemize} \setlength{\itemsep}{-1ex}
\item ``Implications of di-photon excess and future strategies for BSM searches at high-energies (theory/pheno)", by Riccardo Torre  (EPFL);
\item ``BSM at the energy frontier'', by Ashutosh Kotwal  (Duke University);
\item ``Unconventional signatures'', by John Paul Chou (Rutgers University).
\end{itemize}


\noindent The search for physics beyond the standard model is the primary objective of the LHC physics program, as well as particle physics in general, and therefore all avenues for discovery should be exploited in the future.  The experimental necessity of a candidate particle to explain dark matter with a relic density being consistent with it having an electroweak interaction, and the theoretical need of new physics to stabilize the Higgs boson mass at the electroweak scale, both lead to the expectation of new particles to be discovered at the LHC.    New physics can reveal itself as small deviations from parameters of the standard model, such as Higgs boson couplings, as mentioned in section~\ref{higgs}, or deviations in the decay rates of rare processes, as discussed in section~\ref{flavour}.  In these cases, the evidence of new physics is indirect, often generated through higher order loop corrections.   Here, we focus on direct searches for physics beyond the standard model, in which new particles are produced and directly observed.

\medskip 
\noindent Over the course of the current LHC program which ends in $\sim$ 2023 after acquiring 300 fb$^{-1}$, and the HL-LHC program which will acquire 10 times this dataset over a subsequent 10 years of running, physicists have a goal that dark matter will be identified and a means to stabilize the Higgs boson mass will be uncovered, along with other new, unexpected physics.  Future particle colliders beyond the LHC have been proposed and are at various stages of R\& D, including linear colliders with energies ranging from 500 to 3000 GeV center of mass, a circular lepton colliders of 90 - 350 GeV in new 50 - 100 km tunnels, and circular proton colliders of 50 - 100 TeV in these same tunnels.  There is also an option to reuse the LHC tunnel, but replace the current 8.3 T dipole steering magnets with 16 T magnets to approximately double the energy of the LHC collisions.   The primary challenge in higher energy circular machines is achieving these high magnetic fields at low enough cost and production reliability. 
\medskip 

\noindent The current LHC experiments have implemented a comprehensive search strategy for the signatures of supersymmetry, dark matter, extra dimensions, composite Higgs sectors, extended Higgs doublet sectors, additional heavy quarks, and a variety of other models.  These searches, although typically applied to specific models, would uncover a wide range of new physics scenarios.  Searches for new unexpected resonances of leptons, quarks, or bosons would uncover new particles and forces at the TeV scale.  

% mention Riccardo Torre
\medskip
\noindent  At the time of the Swiss strategy workshop, there was a culmination of excitement over the identification of an excess in the diphoton mass spectrum at 750 GeV that had been presented at the end of 2015 by both the CMS and ATLAS $\sqrt{s}=$13 TeV data with 3.4 $\sigma$ (1.6 $\sigma$ ) and 3.9 $\sigma$ (2.1 $\sigma$) local (global) significance, respectively.  The theoretical community had become fully engaged in speculating on the class of models which could produce such an excess given other experimental contraints, and also in predicting other possible new physics that could be expected at the LHC.   While the excess was found not to persist in the 2016 data that was revealed 2 months later, the mobilization of the theoretical community had already produced more than 400 papers on the high energy physics e-print archive (arXiv) to address the implications of this possible signal. 
This signal forced the particle physics community to consider whether the current and future particle physics program was sufficient to address such implications.
%%(this IS the searches session summary  as dicussed in section\ref{bsmsearches}. 
\medskip

% mention Ashutosh Kotwal
\noindent One of the challenges exposed for future higher energy accelerators was the implication on particle identification and measurement. For particles produce with 7 times the energy as the LHC, some combination of a larger calorimeter, improved granularity for more objects being boosted closer together, muon chambers at higher radius, higher detector magnetic fields, a larger dynamic range of measurements, more forward-detection capability, as well as faster detectors, and a higher bandwidth and more intelligent trigger system would be necessary to maintain the same performance for objects at the highest in these future colliders as is for the current LHC on its highest energy objects produced from 13 TeV collisions.   These numerous detector challenges will need to be pursued by the Swiss particle physics community over the upcoming years. 
\medskip

% mention John Paul Chou 
\noindent Another challenge of the LHC and future detectors is to ensure that unconventional, yet highly motivated new physics channels, are not missed due to detector and trigger design.  Long-lived particles arise naturally in a wide range of models, and become more likely due to the absence of evidence for strongly produce new particles. The main challenges are low production rate compared to QCD multi-jet rates, low $P_T$ objects that are difficult to trigger on, and non-standard algorithms for identifying particles decays outside of the standard interaction region inside the central pixel detectors.  To search for such signatures, techniques such as providing dedicated physics streams of events with reduced trigger thresholds and reduced size, as well as specialized reconstruction algorithms must be implemented.  Opportunities arise from high precision timing that future detectors will achieve and track triggers may provide new ways to search for such signals if they are implemented appropriately.  A possible loss in future sensitivity arises from the lack of $dE/dx$ information of future tracker designs, which is envisioned as necessary for reducing data rates, but will reduce the ability of trackers to identify highly ionizing particles. 



\