\section{Session on Future Accelerators}\label{futureacc}{\it editor: L. Rivkin}

\noindent The session of future accelerators consisted of the following three presentations:
\begin{itemize} \setlength{\itemsep}{-1ex}
\item ``Future hadron colliders: accelerator challenges",  by  Bernhard Auchmann  (PSI/CERN); 
\item ``Future lepton colliders: accelerator challenges",  by Terry Garvey   (PSI); 
\item ``Accelerator R\&D towards highest energies" by Rasmus Ischebeck (PSI);  
\end{itemize}



%\fix{the contribution was not received by the deadline}


\noindent The Future Accelerators session provided an overview of the present
plans for future facilities on the high energy frontier. It covered
the FCC hadron collider study, electron positron colliders, mainly
concentrating on the linear colliders ILC and CLIC, and finally
surveyed the present R\&D on very high gradient acceleration.
\medskip

\noindent The superconducting magnets R\&D program towards 16 - 20 Tesla magnetic
fields in the dipole magnets was the main topic of the Bernhard
Auchmann’s talk on the accelerator challenges for future hadron
colliders. The goal of more than doubling the present state-of-the-art
LHC magnets field can no longer be achieved with the NbTi technology
and requires the use of the NbSn3 superconductor material for the 16
Tesla and HTS material for higher field. Present R\&D effort,
spearheaded by CERN, also involves European national laboratories in
France, Italy, Spain and Switzerland, as well as the US DOE high field
magnet program. Possible uses of such high filed magnets for
synchrotron light sources and for medical applications are an
important factor helping to drive this R\&D. Present FCC study aims at
producing a Conceptual Design Report by the end of 2018, including a
first cost estimate of a possible FCC facility.
\medskip 

\noindent Terry Garvey concentrated mainly on the future plans for electron
positron linear colliders, ILC and CLIC. While the superconducting RF
technology based linear colliders are limited to below 50 MeV/m
accelerating gradient and thus aim at energies below 1 TeV, the normal
conducting accelerating structures of CLIC have demonstrated 100 MeV/m
gradients, thus opening the possibility of reaching center-of-mass
collision energies of several TeV. One of the challenges of these
schemes is the high average power consumption and efforts are under
way to improve the energy efficiency of the proposed schemes.
\medskip

\noindent Rasmus Ischebeck provided an overview of the extensive R\&D efforts
around the world on the schemes to exceed present accelerating
gradients by several orders of magnitude, employing laser and plasma
acceleration. Proof-of-principle experiments have demonstrated
accelerating gradients on the order of tens of GV/m. The high
luminosity requirements for the high energy frontier colliders pose
formidable challenges and most of the techniques reviewed still have a
long way to go to demonstrate the required beam
parameters. Nevertheless the present R\&D efforts aim at possible high
gradient accelerators for uses in other accelerator driven fields like
compact synchrotron radiation sources and compact accelerators for
medical applications.
\medskip

\noindent The future accelerator facilities session made clear the need to
increase the R\&D efforts in order to be able to start on a
construction of the next high energy frontier collider in the post
HL-LHC era.




