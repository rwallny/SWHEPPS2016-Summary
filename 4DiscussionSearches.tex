\section{Day 1 Future Collider Searches}\label{section:searches}

The search for physics beyond the standard model is the primary objective of the LHC physics program, as well as particle physics in general, and therefore all avenues for discovery should be exploited in the future.  The experimental necessity of a candidate particle to explain dark matter with a relic density being consistent with it having an electroweak interaction, and the theoretical need of new physics to stabilize the Higgs boson mass at the electroweak scale, both lead to the expectation of new particles to be discovered at the LHC.   

New physics can reveal itself as small deviations from parameters of the standard model, such as Higgs boson couplings, as mentioned in section~\ref{section:higgs}, or deviations in the decay rates of rare processes, as discussed in Secion~\ref{section:bphysics}.  In these cases, the evidence of new physics is indirect, often generated through higher order loop corrections.   Here, we focus on direct searches for physics beyond the standard model, in which new particles are produced and directly observed.

Over the course of the current LHC program which ends in $\sim$ 2023 after acquiring 300 fb$^{-1}$, and the HL-LHC program which will acquire 10 times this dataset over a subsequent 10 years of running, physicists have a goal that dark matter will be identified and a means to stabilize the Higgs boson mass will be uncovered, along with other new, unexpected physics.  Future particle colliders beyond the LHC have been proposed and are at various stages of R\& D, including linear colliders with energies ranging from 500 to 3000 GeV center of mass, a circular lepton colliders of 90 - 350 GeV in new 50 - 100 km tunnels, and circular proton colliders of 50 - 100 TeV in these same tunnels.  There is also an option to reuse the LHC tunnel, but replace the current 8.3 T dipole steering magnets with 16 T magnets to approximately double the energy of the LHC collisions.   The primary challenge in higher energy circular machines is achieving these high magnetic fields at low enough cost and production reliability. 

The current LHC experiments have implemented a comprehensive search strategy for the signatures of supersymmetry, dark matter, extra dimensions, composite Higgs sectors, extended Higgs doublet sectors, additional heavy quarks, and a variety of other models.  These searches, although typically applied to specific models, would uncover a wide range of new physics scenarios.  Searches for new unexpected resonances of leptons, quarks, or bosons would uncover new particles and forces at the TeV scale.  

% mention Riccardo Torre
At the time of the Swiss strategy workshop, there was a culmination of excitement over the identification of an excess in the diphoton mass spectrum at 750 GeV that had been presented at the end of 2015 by both the CMS and ATLAS $\sqrt{s}=$13 TeV data with 3.4 $\sigma$ (1.6 $\sigma$ ) and 3.9 $\sigma$ (2.1 $\sigma$) local (global) significance, respectively.  The theoretical community had become fully engaged in speculating on the class of models which could produce such an excess given other experimental contraints, and also in predicting other possible new physics that could be expected at the LHC.   While the excess was found not to persist in the 2016 data that was revealed 2 months later, the mobilization of the theoretical community had already produced more than 400 papers on the high energy physics e-print archive (arXiv) to address the implications of this possible signal. 
This signal forced the particle physics community to consider whether the current and future particle physics program was sufficient to address such implications, as dicussed in section\ref{section:discussionsearches}. 

% mention Ashutosh Kotwal
One of the challenges exposed for future higher energy accelerators was the implication on particle identification and measurement. For particles produce with 7 times the energy as the LHC, some combination of a larger calorimeter, improved granularity for more objects being boosted closer together, muon chambers at higher radius, higher detector magnetic fields, a larger dynamic range of measurements, more forward-detection capability, as well as faster detectors, and a higher bandwidth and more intelligent trigger system would be necessary to maintain the same performance for objects at the highest in these future colliders as is for the current LHC on its highest energy objects produced from 13 TeV collisions.   These numerous detector challenges will need to be pursued by the Swiss particle physics community over the upcoming years. 

% mention John Paul Chou 
Another challenge of the LHC and future detectors is to ensure that unconventional, yet highly motivated new physics channels, are not missed due to detector and trigger design.  Long-lived particles arise naturally in a wide range of models, and become more likely due to the absence of evidence for strongly produce new particles. The main challenges are low production rate compared to QCD multi-jet rates, low $P_T$ objects that are difficult to trigger on, and non-standard algorithms for identifying particles decays outside of the standard interaction region inside the central pixel detectors.  To search for such signatures, techniques such as providing dedicated physics streams of events with reduced trigger thresholds and reduced size, as well as specialized reconstruction algorithms must be implemented.  Opportunities arise from high precision timing that future detectors will achieve and track triggers may provide new ways to search for such signals if they are implemented appropriately.  A possible loss in future sensitivity arises from the lack of $dE/dx$ information of future tracker designs, which is envisioned as necessary for reducing data rates, but will reduce the ability of trackers to identify highly ionizing particles. 





\section{Day 1 Summary: Discussion of Searches at the LHC}\label{section:discussionsearches}

The discussion was centered around BSM searches at the energy frontier and touched on the physics opportunities, the experimental challenges, and the long-term perspective, with focus on the Swiss perspective.

The various new physics benchmarks that were discussed in the Session on Searches for BSM Physics at the LHC, see Section~\ref{lhcbsm}, illustrated that higher integrated luminosities and the highest possible center-of-mass energy greatly benefit direct searches for new physics.

A discovery in Run 2 or shortly after would allow us to design a more coherent roadmap and would give us ammunition to make a more concrete physics case.  The excess observed in the 2015 data set (but not confirmed with the 2016 data set) in the resonance search around a diphoton invariant mass of 750 GeV was used as an example to hypothesize what an impact such a discovery and the search for potentially associated new particles would have on our overall experimental strategy.  The conclusion was that our current strategy holds in such a case and that the best bet is to continue the tradition of general-purpose detectors based on triggering, tracking and calorimetry capabilities.  Many of the present searches (e.g. for unconventional signatures such as long-lived particles or highly boosted objects, or searches based on partial event readout) had not been anticipated at the time of the design of the LHC detectors.  In order to prepare for a similar unforeseeable development in the future where the detectors would have to be used for unanticipated new signatures, the detectors should be designed with emphasis on redundancy, optimal resolution (energy, spatial, timing), read-out of as much additional information as possible (such as $dE/dx$), and in particular triggering and reconstruction capabilities of unconventional signatures (such as long-lived particles).

Tracking detectors were identified as a particularly promising direction for the Swiss community to invest in, given the extensive expertise on the R\&D, construction and commissioning of the current and future ATLAS and CMS trackers, particularly regarding the pixel technologies.  A clear target is a cost-effective, radiation-hard, low-mass all-pixel tracker with excellent spatial and timing resolution and track triggering capabilities.  Resolution is needed for pile-up rejection in a HL-LHC environment or beyond, for fast pattern recognition, for excellent capabilities of flavor-tagging and reconstruction of long-lived particles, for a high $p_T$ resolution, as well as substructure analysis of hadronically decaying top quarks, W, Z, or Higgs bosons with very high transverse momenta, resulting in collimated jets of particles which cannot be resolved anymore by the calorimeters.  Extension to the forward direction beyond $|\eta |$ of 2.5 is also desirable.

One promising avenue for our community was identified as the energy frontier with the current ultimate goal of a $pp$ collider at a center-of-mass energy of 100 TeV (FCC).  The machine's dipole magnet R\&D is the key driver of the timescale and feasibility of this project.  Investing in this magnet R\&D was identified as an area of interest for our Swiss community.  A $\sim30$ TeV (HE-LHC) could be a stepping stone for this technology.  A discovery of high-mass new physics in Run 2 or 3 (such as the above-mentioned temporary excess at a mass of 750 GeV) would bolster the physics case for a HE-LHC. It should be considered to make sure that the design of the HL-LHC detectors does not result in limitations for a potential HE-LHC scenario.

An important figure of merit for future experiments is to retain maximum sensitivity for a set of ever-growing new physics benchmarks.  A key role in the detector design plays the simulation of the detector configurations, the assessment of the physics opportunites for the simulated new physics benchmarking for the future machines under study, while also taking into account new software developments (such as machine learning or more concretely particle flow).  A close loop needs to be kept where the benchmark analyses inform the detector design by giving feedback on detector requirements.  Development of common tools and collaboration across current LHC experiments should be considered.


\fix{editors: Ben, Tobias}
