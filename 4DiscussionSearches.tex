\section*{Day 1 Summary: Discussion of Searches at the LHC}\label{discussionsearches}{\it discussion leaders: B. Kilminster, T. Golling}


\noindent The discussion was centered around BSM searches at the energy frontier and touched on the physics opportunities, the experimental challenges, and the long-term perspective, with focus on the Swiss perspective.
The various new physics benchmarks that were discussed in the Session on Searches for BSM Physics at the LHC, see section~\ref{lhcbsm}, illustrated that higher integrated luminosities and the highest possible 
center-of-mass energy greatly benefit direct searches for new physics.
\medskip

\noindent  A discovery in Run 2 or shortly after would allow us to design a more coherent roadmap and would give us ammunition to make a more concrete physics case.  The excess observed in the 2015 data set (but not confirmed with the 2016 data set) in the resonance search around a diphoton invariant mass of 750 GeV was used as an example to hypothesize what an impact such a discovery and the search for potentially associated new particles would have on our overall experimental strategy.  The conclusion was that our current strategy holds in such a case and that the best bet is to continue the tradition of general-purpose detectors based on triggering, tracking and calorimetry capabilities.  Many of the present searches (e.g. for unconventional signatures such as long-lived particles or highly boosted objects, or searches based on partial event readout) had not been anticipated at the time of the design of the LHC detectors.  In order to prepare for a similar unforeseeable development in the future where the detectors would have to be used for unanticipated new signatures, the detectors should be designed with emphasis on redundancy, optimal resolution (energy, spatial, timing), read-out of as much additional information as possible (such as $dE/dx$), and in particular triggering and reconstruction capabilities of unconventional signatures (such as long-lived particles).
\medskip

\noindent Tracking detectors were identified as a particularly promising direction for the Swiss community to invest in, given the extensive expertise on the R\&D, construction and commissioning of the current and future ATLAS and CMS trackers, particularly regarding the pixel technologies.  A clear target is a cost-effective, radiation-hard, low-mass all-pixel tracker with excellent spatial and timing resolution and track triggering capabilities.  Resolution is needed for pile-up rejection in a HL-LHC environment or beyond, for fast pattern recognition, for excellent capabilities of flavor-tagging and reconstruction of long-lived particles, for a high $p_T$ resolution, as well as substructure analysis of hadronically decaying top quarks, W, Z, or Higgs bosons with very high transverse momenta, resulting in collimated jets of particles which cannot be resolved anymore by the calorimeters.  Extension to the forward direction beyond $|\eta |$ of 2.5 is also desirable. One promising avenue for our community was identified as the energy frontier with the current ultimate goal of a $pp$ collider at a center-of-mass energy of 100 TeV (FCC).  The machine's dipole magnet R\&D is the key driver of the timescale and feasibility of this project.  Investing in this magnet R\&D was identified as an area of interest for our Swiss community.  A $\sim30$ TeV (HE-LHC) could be a stepping stone for this technology.  A discovery of high-mass new physics in Run 2 or 3 (such as the above-mentioned temporary excess at a mass of 750 GeV) would bolster the physics case for a HE-LHC. It should be considered to make sure that the design of the HL-LHC detectors does not result in limitations for a potential HE-LHC scenario.
\medskip

\noindent An important figure of merit for future experiments is to retain maximum sensitivity for a set of ever-growing new physics benchmarks.  A key role in the detector design plays the simulation of the detector configurations, the assessment of the physics opportunites for the simulated new physics benchmarking for the future machines under study, while also taking into account new software developments (such as machine learning or more concretely particle flow).  A close loop needs to be kept where the benchmark analyses inform the detector design by giving feedback on detector requirements.  Development of common tools and collaboration across current LHC experiments should be considered.
