\section{Summary of the Workshop}\label{summary}
{\it discussion leader: T. Nakada}

\noindent The Discussion Session on the future accelerators took place during the last day of the workshop. In the discussion, opinions were exchanged based on the current situation of physics and accelerator R\&D and design studies presented throughout the Workshop.
\medskip

\noindent There are several compelling observations that demonstrate the existence of physics beyond the Standard Model: non-zero mass of neutrinos, existence of dark matter and dark energy in the universe and the abundance of matter observed in the universe. Motivated by various theoretical considerations, it has been generally thought that energy threshold for new physics could be at around 1 TeV, i.e. accessible by the LHC. On the other hand, experiments with the energy frontier machines and precision experiments at various lower energies have not been able to establish an ambiguous effect of physics beyond the Standard model so far. Observed “anomalies” are either statistically not yet significant or suffer from systematic uncertainties. As a result, we have little idea where the energy scale for the new physics is. 
\medskip

\noindent Since there is little idea where the new physics lies, searches must be carried out at both energy and precision frontiers. As being done for the other particles, properties for the Higgs particle should be studied in precision to look for a deviation from the Standard Model predictions. An important question to be addressed is whether we need a dedicated machine, in addition to the LHC and its luminosity upgrade, and if yes, what kind of machine. Complementarity to the LHC calls for a lepton machine for its clean environment: the ILC, CLIC, FCC-ee, CEPC and a Muon-collider are being considered and are  at various stages of development. It might be also advantageous to have a lepton machine for the top quark precision study. From the scale of machine, construction of some of those machines could be undertaken with a regional initiative with a large international participation. There is no clear motivation for a lepton collider that can reach a  centre of mass energies beyond 1 TeV for those precision studies.
\medskip

\noindent Hadron machines have traditionally been considered as a discovery machine where recent history shows that a factor of about ten increases in the centre of mass energies led to discoveries of new particles: notable examples have been the  W and Z at SppS, the top quark at the Tevatron and the Higgs boson at the LHC, where the costs of machines have also progressively increased. Although running at high luminosities increases the energy reach, the energy is the key for the future hadron machines. Fcc-pp and SPPC are aiming at a centre of mass energy of 100 TeV, which should access the energy scale of 10 TeV. Given the very high cost associated for those machines, an expected level of guaranteed physics success from the public, policy makers and funding agencies could be very high for constructing such accelerators. Required cost and human resources required for such a project demands that a project must be initiated as a global project from the beginning among equally partners. Are we ready for this? 

\medskip 

\noindent It has been well established that for deciding the next energy frontier machine, physics inputs from precision experiment at low energies are essential. Unlike the energy frontier machines, facilities for those experiments can be well accommodated by national laboratories, while the experiments are always operated by an international collaboration.  After the High Luminosity LHC, CLIC with the low energy phase-1 and High Energy LHC upgrade appears to be the two most probable options for the future energy frontier machine. Participation in the ILC and/or CEPC depends solely on whether the Japanese or Chinese government, respectively, will make a firm declaration of intention to construct/host such machines. Muon colliders still need substantial R\&D and without physics motivation far above 1 TeV, it would be difficult to justify it. While a Higgs factory can still be somewhat independently motivated, the LHC Run-2 results will have a big impact in the decision of the future high energy frontier machines. 
\medskip

\noindent What is commonly understood is that  accelerator R\&D are essential, for concrete accelerator studies and for more generic technological issues, including particle sources. It has an advantage that most of the technologies can be used in many other fields. This also apply for the detector R\&D.  In conclusion, we need to continue our discussion in a regular interval in the coming future as a part of the preparation for producing the Swiss Road Map for Particle and Astropartice physics in 2018.

