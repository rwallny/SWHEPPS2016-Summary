\section{Outlook}\label{outlook}{\it editors: T. Nakada}

\noindent The European particle physics programme is clearly driven by the European Strategy for Particle Physics, where the current version was approved by CERN Council in May 2013. Since then, the HL-LHC project was approved and started at CERN and the neutrino platform has also been constructed at CERN. While the CLIC R\&D continues to make advancements, a new study for a very large circular machine, FCC, has started, where  both machines are an option for a future CERN energy frontier machine. Those three activities are among the four highest priorities given in the Strategy. The fourth one, ILC, depends on the decision by the Japanese Government. \medskip

\noindent The next European Strategy update is now anticipated in spring 2020. This means that a European wide community discussion will take place during 2019 and CHIPP must be prepared for this discussion. The Swiss high energy frontier group must now prepare a white paper, which should be an input for the new Swiss Roadmap for particle and astroparticle physics in 2018, which in turn should be the Swiss input to the European Strategy discussion in 2019. 
While the scale of the required infrastructure, cost, timescale, logistic complexity and social structure are very different between energy frontier experiments and precision physics experiments at lower energies, their complementarity in physics in the quest for the search of phenomena beyond the Standard Model has started to be well recognised. This became evident also in this workshop. For identifying and justifying the next high-energy frontier machines to be built, information from the both frontiers will be needed. Therefore, it is vital that the two community work together to establish the Swiss priority for the next machine and experiments.  \medskip 
  
\noindent For the Swiss Roadmap discussion in 2018, the LHC results from the 2017 data, including the ones from LHCb would be essential, together with the low energy precision data. The status of the high field magnet R\&D might indicate a timescale and cost for a possible intermediate machine, HE-LHC. The cost estimates for the FCC as well as the cost and physics potential of the new baseline 380~GeV version of the CLIC are another important key issues. Status of the Japanese attempt to start a 250 GeV ILC with a later energy upgrade plan and a Chinese plan for the CEPC circular e+e- collider must also be followed in the Swiss discussion. 



